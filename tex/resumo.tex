\begin{resumo}
As mudanças climáticas representam um dos principais desafios científicos e sociais do século XXI, com evidências consistentes de alterações nos padrões de temperatura, precipitação e radiação solar em diferentes regiões do planeta. Este trabalho investigou as tendências dessas variáveis climáticas nos estados de São Paulo e Rio Grande do Norte durante o período de 1990 a 2024, utilizando dados observacionais de estações meteorológicas do INMET e ANA (2004–2024), integrados a produtos de satélite (CHIRPS, GOES-16, CLARA-A3) e reanálise (ERA5-Land). A validação entre as diferentes fontes foi realizada por meio de regressão ortogonal, evidenciando concordância excepcional para temperatura ($R \geq 0{,}9$), boa a muito boa para radiação solar ($R = 0{,}6$ a $0{,}9$) e moderada a boa para precipitação ($R = 0{,}7$ a $0{,}9$), embora esta última apresente maiores incertezas associadas à sua elevada variabilidade espaço-temporal.
As tendências climáticas foram detectadas pelo teste de Mann-Kendall e quantificadas pelo estimador de Sen, com análise sazonal. Os resultados evidenciaram aquecimento generalizado em ambos os estados, com maior magnitude em São Paulo, onde o aumento acumulado variou entre aproximadamente $0{,}6$ e $2{,}5$ °C, enquanto no Rio Grande do Norte situou-se entre $0{,}3$ e $0{,}6$ °C. Esse aquecimento foi consistente com tendências predominantemente positivas da radiação solar incidente. A precipitação apresentou comportamento altamente variável, com tendências predominantemente negativas em São Paulo, porém com baixa significância estatística, enquanto no Rio Grande do Norte a elevada variabilidade interanual dificultou a identificação de tendências com significância estatística consistente.
A análise quantitativa da relação entre cobertura da terra e variáveis climáticas, com base nos dados do MapBiomas, demonstrou que a urbanização constitui o principal fator antrópico associado às alterações térmicas em escala local. Em São Paulo, aproximadamente 93{,}8\% dos pontos apresentaram associação estatisticamente significativa entre urbanização e temperatura, com coeficiente mediano de regressão $\beta_{\mathrm{med}} = 4{,}75$, evidenciando elevada sensibilidade térmica e caracterizando claramente o efeito de ilha de calor urbana. Em contraste, no Rio Grande do Norte, apenas cerca de 2{,}0\% dos pontos apresentaram associação significativa, refletindo a limitação hídrica característica do clima semiárido. As formações vegetais nativas apresentaram influência mais limitada, especialmente no Rio Grande do Norte, demonstrando pouca relação entre a diminuição da formação savânica como influência direta às variáveis climáticas, com a maior significância sendo de 39{,}5\% em relação à radiaçõ solar. Já em São Paulo os resultados indicaram mesmo que sutilmente um comportamento modulador da formação florestal sobre a temperatura e radiação solar, com significância de 68{,}7\% para temperatura e 55{,}2\% para radiação.
O estudo também documentou transformações significativas no uso e cobertura da terra, incluindo redução expressiva das áreas de pastagem em São Paulo (de 37{,}4\% para 22{,}1\%) e redução da formação savânica no Rio Grande do Norte (de 27{,}0\% para 23{,}6\%). De forma geral, os resultados demonstram que as mudanças observadas resultam da interação entre forçantes climáticas globais e transformações locais da superfície terrestre, com a urbanização atuando como importante modulador regional das tendências térmicas.

\vspace{0.5cm}
\noindent\textbf{Palavras-chave:} Mudanças climáticas; Tendências climáticas; cobertura da terra; Ilha de calor urbana; São Paulo; Rio Grande do Norte; Temperatura; Precipitação; Radiação Solar.
\end{resumo}