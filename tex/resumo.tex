\begin{resumo}
	As mudanças climáticas representam um dos principais desafios científicos e sociais do século XXI, com evidências claras de alterações nos padrões de temperatura, precipitação e radiação solar em diferentes regiões do planeta. Este trabalho investigou as tendências de variáveis meteorológicas nos estados de São Paulo e Rio Grande do Norte durante o período de 1990 a 2024, utilizando dados observacionais de estações meteorológicas do INMET e ANA (2004-2024) integrados a produtos de satélite (CHIRPS, GOES-16, CLARA-A3) e reanálise (ERA5-Land). A validação via correlação entre fontes foi realizada por meio de regressão ortogonal, demonstrando concordância excepcional para temperatura ($R \geq 0{,}9$) e boa a muito boa para radiação solar ($R = 0{,}6$ a $0{,}9$), enquanto a precipitação apresentou incertezas maiores apesar de correlações moderadas a boas ($R = 0{,}7$ a $0{,}9$). As tendências climáticas foram detectadas pelo teste de Mann-Kendall e quantificadas pelo estimador de Sen, com análises sazonais para verão (DJF), outono (MAM), inverno (JJA) e primavera (SON). Os resultados evidenciam aquecimento generalizado em ambos os estados, com São Paulo apresentando aumento de 1{,}0 a 1{,}5 °C e Rio Grande do Norte de 0{,}3 a 0{,}6 °C ao longo das três décadas e meia analisadas, sendo a primavera a estação com aquecimento mais intenso. A precipitação apresentou padrões altamente heterogêneos em ambos os estados, com tendências predominantemente negativas em São Paulo mas baixa significância estatística devido à alta variabilidade interanual, enquanto no Rio Grande do Norte a variabilidade extrema (coeficiente de variação $>40$ a 50\%) mascarou completamente possíveis sinais sistemáticos. A radiação solar mostrou tendências predominantemente positivas em São Paulo, particularmente durante o inverno, e padrões mais heterogêneos no Rio Grande do Norte. A análise quantitativa da relação entre cobertura da terra (dados MapBiomas) e variáveis climáticas revelou que a urbanização constitui a principal forçante climática detectável, com 93{,}8\% dos pontos em São Paulo apresentando relação significativa entre área urbanizada e temperatura (sensibilidade mediana de 0{,}97 °C por ponto percentual de urbanização), implicando aquecimento de 3 a 5 °C em áreas intensamente urbanizadas. Este efeito mostrou-se fortemente modulado pelo contexto climático, com apenas 2{,}0\% de significância no Rio Grande do Norte devido ao menor contraste evapotranspirativo entre áreas urbanas e rurais em clima semiárido. As formações vegetais nativas (Mata Atlântica em São Paulo e Caatinga no Rio Grande do Norte) apresentaram relações fracas e espacialmente inconsistentes com as variáveis climáticas, refletindo primariamente distribuição geográfica das vegetações remanescentes ao invés de efeitos causais diretos. O estudo documentou transformações substanciais na cobertura da terra, incluindo conversão de pastagens (37{,}4\% para 22{,}1\% do território) em agricultura intensiva em São Paulo e degradação severa da Caatinga no Rio Grande do Norte, com redução de 79\% para 35{,}5\% em áreas de maior pressão antrópica entre 1990 e 2020.
	
	\vspace{0.5cm}
	\noindent\textbf{Palavras-chave:} Mudanças climáticas; Tendências climáticas; Uso e cobertura da terra; Ilha de calor urbana; Validação de dados de satélite; São Paulo; Rio Grande do Norte.
\end{resumo}