\begin{abstract}
Climate change represents one of the main scientific and societal challenges of the 21st century, with consistent evidence of alterations in temperature, precipitation, and solar radiation patterns across different regions of the planet. This study investigated trends in these climatic variables in the states of São Paulo and Rio Grande do Norte, Brazil, during the period 1990–2024, using observational data from meteorological stations of INMET and ANA (2004–2024), integrated with satellite products (CHIRPS, GOES-16, CLARA-A3) and reanalysis data (ERA5-Land). Cross-validation among the different data sources was performed using orthogonal regression, demonstrating exceptional agreement for temperature ($R \geq 0.9$), good to very good agreement for solar radiation ($R = 0.6$ to $0.9$), and moderate to good agreement for precipitation ($R = 0.7$ to $0.9$), although the latter presented greater uncertainties associated with its high spatial and temporal variability.
Climate trends were detected using the Mann–Kendall test and quantified using Sen’s slope estimator, with seasonal analysis. The results revealed widespread warming in both states, with greater magnitude in São Paulo, where the accumulated increase ranged from approximately $0.6$ to $2.5$ °C, while in Rio Grande do Norte the increase ranged between $0.3$ and $0.6$ °C. This warming was consistent with predominantly positive trends in incident solar radiation. Precipitation exhibited highly variable behavior, with predominantly negative trends in São Paulo, although with low statistical significance, while in Rio Grande do Norte the high interannual variability hindered the identification of statistically robust trends.
The quantitative analysis of the relationship between land cover and climatic variables, based on MapBiomas data, demonstrated that urbanization constitutes the main anthropogenic factor associated with thermal changes at the local scale. In São Paulo, approximately 93.8\% of the analyzed points showed statistically significant association between urbanization and temperature, with a median regression coefficient of $\beta_{\mathrm{med}} = 4.75$, indicating high thermal sensitivity and clearly characterizing the urban heat island effect. In contrast, in Rio Grande do Norte, only about 2.0\% of the points showed significant association, reflecting the hydrological limitations characteristic of semiarid climates. Native vegetation formations showed more limited influence, particularly in Rio Grande do Norte, where the reduction of savanna formation exhibited weak relationships with climatic variables, with maximum statistical significance of 39.5\% for solar radiation. In São Paulo, forest formation exhibited a subtle but consistent modulating effect on temperature and solar radiation, with statistical significance of 68.7\% and 55.2\%, respectively.
The study also documented significant land use and land cover changes, including substantial reduction of pasture areas in São Paulo (from 37.4\% to 22.1\%) and reduction of savanna formation in Rio Grande do Norte (from 27.0\% to 23.6\%). Overall, the results demonstrate that the observed climatic changes result from the interaction between global climate forcing and local land surface transformations, with urbanization acting as an important regional modulator of thermal trends.

\vspace{0.5cm}
\noindent\textbf{Keywords:} Climate change; Climate trends; Land cover; Urban heat island; São Paulo; Rio Grande do Norte; Temperature; Precipitation; Solar radiation.
\end{abstract}