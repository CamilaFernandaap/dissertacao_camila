\begin{abstract}
	Climate change represents one of the main scientific and social challenges of the 21st century, with clear evidence of changes in temperature, precipitation, and solar radiation patterns across different regions of the planet. This work sought to present different results for meteorological variable trends in the states of São Paulo and Rio Grande do Norte during the period from 1990 to 2024, using observational data from INMET and ANA meteorological stations (2004--2024) integrated with satellite products (CHIRPS, GOES-16, CLARA-A3) and reanalysis (ERA5-Land). Correlation-based validation between sources was performed through orthogonal regression, demonstrating exceptional agreement for temperature ($R \geq 0.9$) and fair to very good for solar radiation ($R = 0.6$ to $0.9$), while precipitation showed greater uncertainties despite moderate to good correlations ($R = 0.7$ to $0.9$). Climate trends were detected by the Mann-Kendall test and quantified by Sen's slope estimator, with seasonal analyses for summer (DJF), autumn (MAM), winter (JJA), and spring (SON). Results show widespread warming in both states, with São Paulo showing an increase of 1.0 to 1.5 °C and Rio Grande do Norte of 0.7 to 1.1 °C over the three and a half decades analyzed, with spring being the season with the most intense warming. Precipitation showed highly heterogeneous patterns in both states, with predominantly negative trends in São Paulo but low statistical significance due to high interannual variability, while in Rio Grande do Norte extreme variability (coefficient of variation $>40$ to 50\%) completely masked possible systematic signals. Solar radiation showed predominantly positive trends in São Paulo, particularly during winter, and more heterogeneous patterns in Rio Grande do Norte. Quantitative analysis of the relationship between land cover (MapBiomas data) and climate variables revealed that urbanization constitutes the main detectable climate forcing, with 93.8\% of points in São Paulo showing a significant relationship between urbanized area and temperature (median sensitivity of 0.97 °C per percentage point of urbanization), implying warming of 3 to 5 °C in intensely urbanized areas. This effect was strongly modulated by climatic context, with only 2.0\% significance in Rio Grande do Norte due to lower evapotranspirative contrast between urban and rural areas in semiarid climate. Native vegetation formations (Atlantic Forest in São Paulo and Caatinga in Rio Grande do Norte) showed weak and spatially inconsistent relationships with climate variables, primarily reflecting the geographical distribution of remnant vegetation rather than direct causal effects. The study also addressed substantial transformations in land cover, including massive conversion of pastures to intensive agriculture in São Paulo and severe degradation of Caatinga in Rio Grande do Norte (reduction from 79\% to 35.5\% in areas under greater anthropogenic pressure between 1990 and 2020).
	
	\vspace{0.5cm}
	\noindent\textbf{Keywords:} Climate change; Climate trends; Land use and land cover; Urban heat island; Satellite data validation; São Paulo; Rio Grande do Norte.
\end{abstract}