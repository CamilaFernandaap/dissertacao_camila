\chapter{Análise Integrada de Variáveis Climáticas: Tendências (Mann-Kendall), Séries Filtradas (1-2-1) e Anomalias por Localidade}
\label{ap01}

Considerando a análise detalhada das tendências climáticas para localidades representativas dos estados de São Paulo e Rio Grande do Norte no período de 1990 a 2024. Os gráficos foram organizados em três colunas para cada variável climática (temperatura, radiação solar e precipitação):

\begin{itemize}
	\item \textbf{Coluna esquerda}: Série histórica completa (1990--2024) com dados anuais (linha vermelha/amarela/azul), tendência linear obtida pelo teste de Mann--Kendall (linha tracejada preta) e significância estatística indicada quando $p < 0{,}05$.
	
	\item \textbf{Coluna central}: Série filtrada pelo método de média móvel ponderada 1-2-1 (10 anos), destacando a variabilidade de médio prazo. A área sombreada indica a diferença entre a série filtrada e a média climatológica (linha pontilhada azul).
	
	\item \textbf{Coluna direita}: Anomalias anuais em relação à média climatológica, diferenciando valores acima da média (barras vermelhas) e abaixo da média (barras azuis), calculadas para cada estação do ano.
\end{itemize}

As análises são apresentadas separadamente para as quatro estações climáticas: verão austral (DJF -- dezembro, janeiro, fevereiro), outono (MAM -- março, abril, maio), inverno (JJA -- junho, julho, agosto) e primavera (SON -- setembro, outubro, novembro).

Neste apêndice, apresentam-se os resultados para \textbf{São Paulo (capital)} e \textbf{Parnamirim (RN)} como exemplos representativos das regiões metropolitanas dos estados analisados. As figuras completas para todas as demais localidades mencionadas nas Tabelas~\ref{tab:tendencias_temp_sp_a}, \ref{tab:tendencias_temp_sp_b}, \ref{tab:tendencias_radiacao_sp_a}, \ref{tab:tendencias_radiacao_sp_b}, \ref{tab:tendencias_prec_sp_a}, \ref{tab:tendencias_prec_sp_b}, \ref{tab:tendencias_temp_rn_a}, \ref{tab:tendencias_temp_rn_b}, \ref{tab:tendencias_rad_rn_a}, \ref{tab:tendencias_rad_rn_b}, \ref{tab:tendencias_prec_rn_a} e \ref{tab:tendencias_prec_rn_b}  estão disponíveis no repositório do projeto: \url{https://github.com/CamilaFernandaap/resultados/issues/1}\footnote{Os dados brutos, scripts de análise e gráficos completos para todas as localidades analisadas estão disponíveis sob licença aberta no repositório GitHub do projeto.}.

\subsection*{São Paulo - Capital}

\begin{figure}[H]
	\centering
	\includegraphics[width=\textwidth]{Analise_Integrada_10_Capital_São_Paulo_DJF.png}
	\caption{Análise integrada de variáveis climáticas para São Paulo (capital) -- Verão Austral (DJF), 1990-2024. Superior: temperatura; Centro: radiação solar; Inferior: precipitação.}
	\label{fig:apendice_sao_paulo_djf}
\end{figure}

\begin{figure}[H]
	\centering
	\includegraphics[width=\textwidth]{Analise_Integrada_10_Capital_São_Paulo_MAM.png}
	\caption{Análise integrada de variáveis climáticas para São Paulo (capital) -- Outono (MAM), 1990-2024. Superior: temperatura; Centro: radiação solar; Inferior: precipitação.}
	\label{fig:apendice_sao_paulo_mam}
\end{figure}

\begin{figure}[H]
	\centering
	\includegraphics[width=\textwidth]{Analise_Integrada_10_Capital_São_Paulo_JJA.png}
	\caption{Análise integrada de variáveis climáticas para São Paulo (capital) -- Inverno (JJA), 1990-2024. Superior: temperatura; Centro: radiação solar; Inferior: precipitação.}
	\label{fig:apendice_sao_paulo_jja}
\end{figure}

\begin{figure}[H]
	\centering
	\includegraphics[width=\textwidth]{Analise_Integrada_10_Capital_São_Paulo_SON.png}
	\caption{Análise integrada de variáveis climáticas para São Paulo (capital) -- Primavera (SON), 1990-2024. Superior: temperatura; Centro: radiação solar; Inferior: precipitação.}
	\label{fig:apendice_sao_paulo_son}
\end{figure}

\subsection*{Parnamirim - Rio Grande do Norte}

\begin{figure}[H]
	\centering
	\includegraphics[width=\textwidth]{Analise_Integrada_06_Capital_Parnamirim_DJF.png}
	\caption{Análise integrada de variáveis climáticas para Parnamirim (RN) -- Verão Austral (DJF), 1990-2024. Superior: temperatura; Centro: radiação solar; Inferior: precipitação.}
	\label{fig:apendice_parnamirim_djf}
\end{figure}

\begin{figure}[H]
	\centering
	\includegraphics[width=\textwidth]{Analise_Integrada_06_Capital_Parnamirim_MAM.png}
	\caption{Análise integrada de variáveis climáticas para Parnamirim (RN) -- Outono (MAM), 1990-2024. Superior: temperatura; Centro: radiação solar; Inferior: precipitação.}
	\label{fig:apendice_parnamirim_mam}
\end{figure}

\begin{figure}[H]
	\centering
	\includegraphics[width=\textwidth]{Analise_Integrada_06_Capital_Parnamirim_JJA.png}
	\caption{Análise integrada de variáveis climáticas para Parnamirim (RN) -- Inverno (JJA), 1990-2024. Superior: temperatura; Centro: radiação solar; Inferior: precipitação.}
	\label{fig:apendice_parnamirim_jja}
\end{figure}

\begin{figure}[H]
	\centering
	\includegraphics[width=\textwidth]{Analise_Integrada_06_Capital_Parnamirim_SON.png}
	\caption{Análise integrada de variáveis climáticas para Parnamirim (RN) -- Primavera (SON), 1990-2024. Superior: temperatura; Centro: radiação solar; Inferior: precipitação.}
	\label{fig:apendice_parnamirim_son}
\end{figure}