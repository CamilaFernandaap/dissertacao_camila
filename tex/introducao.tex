\chapter{Introdução}
\label{intro}

Nas últimas décadas, observa-se no cotidiano, nos meios de comunicação e no debate público uma crescente preocupação com as mudanças climáticas e seus impactos em diversas áreas, como agricultura, recursos hídricos e saúde pública, fomentando discussões em diferentes esferas da sociedade. Esse cenário tem impulsionado debates entre governos, instituições científicas e a sociedade civil sobre a urgência e a natureza das ações necessárias para mitigação e adaptação aos impactos climáticos, como por exemplo recentemente pela realização da COP30 em Belém, no Pará, que reuniu representantes governamentais, organismos multilaterais, ativistas ambientais, lideranças indígenas e cientistas de diversas áreas do conhecimento.

Ao mesmo tempo, observa-se a disseminação de discursos que questionam ou distorcem evidências científicas consolidadas, especialmente em ambientes digitais, contribuindo para a erosão da confiança pública na ciência. \cite{essien2025} demonstra que a desinformação climática não se limita à negação de evidências físicas, mas configura um fenômeno estrutural de caráter epistemológico, associado a dinâmicas de pós-verdade, afetando a percepção social do risco climático, a formulação de políticas públicas e a legitimidade do conhecimento científico.

Diante desse contexto, torna-se cada vez mais necessário um processo sistemático de compreensão, análise e avanço científico, capaz de identificar tendências, avaliar incertezas e subsidiar estratégias de mitigação e adaptação frente às mudanças climáticas já observadas e às projeções futuras, conforme recomendado pela World Meteorological Organization \citep{wmo2025}, cujo papel institucional inclui a promoção e a articulação da cooperação científica entre as nações-membro para o desenvolvimento de respostas rápidas às mudanças climáticas.

O Painel Intergovernamental sobre Mudanças Climáticas (IPCC) tem documentado de forma consistente evidências inequívocas do aquecimento do sistema climático, indicando que muitas das mudanças observadas desde a década de 1950 não têm precedentes em séculos ou milênios \citep{ipcc2021}. A temperatura média global da superfície aumentou aproximadamente 1,1~°C em relação ao período pré-industrial, com projeções indicando aquecimento adicional nas próximas décadas caso as emissões de gases de efeito estufa se mantenham nos níveis atuais.

\begin{figure}[H]
\centering
\includegraphics[width=0.8\textwidth]{fig/IPCC_AR6_WGI_Figure_4_2.png}
\caption[Mudanças globais projetadas para a temperatura média da superfície e precipitação continental.]{Mudanças globais projetadas para a temperatura média da superfície, precipitação continental, área de gelo marinho do Ártico e nível médio do mar sob diferentes cenários socioeconômicos (SSPs). Fonte: IPCC (2021).}
\label{fig:ipcc_ssp}
\end{figure}

As projeções climáticas globais indicam que a magnitude das mudanças futuras depende fortemente dos cenários de emissões adotados. A Figura~\ref{fig:ipcc_ssp} sintetiza as projeções do Sexto Relatório de Avaliação do IPCC, evidenciando aumentos claros da temperatura média global, alterações nos padrões de precipitação continental, redução da área de gelo marinho do Ártico e elevação do nível médio do mar ao longo do século XXI, sob diferentes trajetórias socioeconômicas (SSPs).

No contexto brasileiro, a diversidade climática associada à grande extensão territorial configura um ambiente propício para estudos de variabilidade e tendências climáticas. O país apresenta regimes climáticos distintos, desde o clima tropical úmido da Amazônia até regiões semiáridas no Nordeste e clima subtropical no Sul, cada qual com vulnerabilidades específicas frente às mudanças climáticas.

\section{Importância das Variáveis Climáticas Analisadas}
Neste estudo, foram analisadas três variáveis meteorológicas: temperatura do ar, precipitação e radiação solar incidente na superfície. Essas variáveis representam componentes essenciais do balanço de energia e do ciclo hidrológico, permitindo caracterizar os regimes climáticos regionais e identificar tendências associadas 
às mudanças climáticas nos estados de São Paulo e Rio Grande do Norte durante, especificamente no período de 1990 a 2025, no qual se concentra este trabalho.

A escolha das variáveis temperatura, precipitação e radiação solar incidente na superfície fundamenta-se em sua relevância como indicadores centrais dos processos climáticos, uma vez que respondem diretamente às forçantes radiativas naturais e antropogênicas \citep{funk2015, wang2008}. Essas variáveis constituem pilares fundamentais do sistema climático terrestre e apresentam elevada sensibilidade a alterações na composição atmosférica.

\subsection{Temperatura}

A temperatura do ar representa o indicador mais direto do aquecimento global, refletindo o balanço energético da atmosfera. Variações na temperatura média influenciam padrões de evapotranspiração, demanda energética, circulação atmosférica e a distribuição espacial de espécies sensíveis ao clima. Eventos extremos de temperatura, como ondas de calor e frio, têm se tornado mais frequentes, com impactos diretos sobre a saúde pública, espécies sensíveis da fauna e flora, e os sistemas urbanos, agrícolas e ecológicos.

Estudos recentes documentam tendências de aquecimento em diversas regiões do Brasil. \cite{abreu2019} investigaram os principais fatores que controlam a temperatura do ar na região Sudeste, identificando a influência da zonalidade (distribuição latitudinal associada à incidência solar), continentalidade (distância em relação ao oceano) e altitude, além da cobertura da terra representada pelo NDVI (Normalized Difference Vegetation Index, ou Índice de Vegetação por Diferença Normalizada) e da cobertura de nuvens. Utilizando 26 anos de dados de 52 estações meteorológicas, os autores observaram um gradiente térmico noroeste--sudeste no estado de São Paulo, com variações de aproximadamente 3~°C na temperatura máxima média e 2~°C na temperatura mínima média.

No contexto urbano, o fenômeno das ilhas de calor urbanas (ICU) intensifica as temperaturas locais, sobretudo em grandes centros urbanos. Campelo (2024) identificou em seu trabalho um aumento da intensidade da ilha de calor urbana de superfície na região metropolitana de São Paulo da ordem de 0,13~°C por década, com intensidades superiores a 1,6~°C durante o dia e 2,6~°C durante a noite.

\subsection{Precipitação}

A precipitação constitui o principal componente do ciclo hidrológico, determinando a disponibilidade de recursos hídricos para abastecimento urbano, agricultura e geração de energia hidrelétrica. Alterações nos regimes pluviométricos podem intensificar tanto eventos de seca quanto de precipitação extrema, impactando diretamente a segurança hídrica e alimentar \citep{funk2015}.

O Brasil apresenta elevada variabilidade espacial e temporal da precipitação. A região Nordeste, caracterizada pelo clima semiárido, é particularmente vulnerável às variações pluviométricas, com impactos diretos sobre a agricultura de subsistência e o abastecimento de água. \cite{marengo2017} analisaram a seca prolongada de 2012--2017, demonstrando que o evento resultou da combinação de anomalias da temperatura da superfície do mar no Atlântico tropical e no Pacífico, afetando os sistemas atmosféricos responsáveis pelo transporte de umidade para a região.

Na região Sudeste, \cite{coelho2016} realizaram um diagnóstico detalhado do evento seco excepcional ocorrido durante o verão austral de 2013--2014 no estado de São Paulo, caracterizando-o como o mais severo da série histórica analisada (1961--2015). O verão de 2014--2015 também apresentou déficits pluviométricos relevantes, porém de menor magnitude. Os autores associaram o evento à redução da atuação da Zona de Convergência do Atlântico Sul e ao término anormalmente precoce da estação chuvosa.

\subsection{Radiação Solar Incidente na Superfície}

A radiação solar global constitui a principal fonte de energia do sistema climático terrestre, controlando processos como o aquecimento da superfície, a evaporação e a fotossíntese. A quantidade de energia solar que efetivamente atinge a superfície resulta da modulação da radiação ao longo de sua trajetória pela atmosfera, sendo fortemente influenciada pela presença de nuvens, aerossóis, vapor d’água e gases atmosféricos, os quais alteram os processos de absorção e espalhamento da radiação solar. Dessa forma, variações na radiação incidente afetam diretamente os balanços energéticos regionais e desempenham papel central na variabilidade climática.

\cite{wang2008} apresentaram uma revisão detalhada das metodologias para a estimativa da irradiância solar incidente na superfície (Surface Solar Irradiance – SSI) a partir de observações por satélite. A Figura \ref{fig:evolucao} ilustra a evolução temporal dessas técnicas desde a década de 1960, destacando marcos históricos no desenvolvimento de sensores orbitais e modelos de transferência radiativa. Os autores demonstram que as primeiras tentativas de estimativa da irradiância solar por sensoriamento remoto remontam à década de 1960 e que, ao longo das décadas seguintes, o avanço dos sensores orbitais e dos modelos de transferência radiativa consolidou os produtos de SSI como a principal ferramenta para o mapeamento da variabilidade espaço-temporal da radiação solar na superfície terrestre.

\begin{figure}[H]
    \centering
    \includegraphics[width=1\linewidth]{fig/evolucao.png}
    \caption[Linha do tempo do desenvolvimento das estimativas de irradiância solar por satélite.]
{Linha do tempo do desenvolvimento das estimativas de irradiância solar de superfície a partir de observações por satélite, abrangendo o período pré-satélite (antes de 1960) até a atualidade. A figura descreve momentos históricos, avanços metodológicos, projetos internacionais (ISCCP, ERBS, LUT) e o surgimento de produtos 
operacionais globais de nova geração (CERES, MODIS, GEWEX SRB). Extraído de \cite{wang2008}.}
    \label{fig:evolucao}
\end{figure}

No contexto operacional, \cite{laszlo2020} descreveram o produto Enterprise Shortwave Radiation Budget da série GOES, que fornece estimativas contínuas da radiação solar em alta resolução temporal e espacial. Esses produtos representam uma ferramenta essencial para estudos regionais de variabilidade e tendências climáticas, permitindo a análise consistente da radiação solar incidente na superfície em diferentes escalas temporais e espaciais.

\begin{figure}[H]
\centering
\includegraphics[width=0.7\textwidth]{fig/G19_ABI_20250621_1700_SWRadiation.jpg}
\caption[Distribuição global da radiação solar e seus componentes.]
{Distribuição global instantânea (às 17 UTC) da radiação fotossinteticamente 
ativa (Photosynthetically Active Radiation), da radiação solar incidente na superfície 
(Downward SW Radiation at surface) e da radiação refletida no topo da atmosfera 
(Reflected SW Radiation at TOA), conforme estimativas derivadas do sistema GOES. . Extraído de: \url{https://www.star.nesdis.noaa.gov/goesr/product_sw.php}}
\label{fig:radiation_components}
\end{figure}

A distribuição espacial da radiação solar e de seus diferentes componentes pode ser 
analisada de forma integrada a partir de observações por satélite, possibilitando a 
caracterização simultânea da energia disponível na superfície, da fração refletida 
no topo da atmosfera e dos fluxos modulados pela cobertura de nuvens e pela presença 
de aerossóis. A Figura~\ref{fig:radiation_components} apresenta valores instantâneos 
estimados para as 17 UTC (14h no horário local de Brasília) de três componentes 
radiométricos derivados do sistema GOES: (i) a radiação fotossinteticamente ativa 
(Photosynthetically Active Radiation - PAR), correspondente à faixa espectral de 
400-700 nm utilizada pelos organismos fotossintetizantes, com valores típicos entre 
0 e 500 W m$^{-2}$; (ii) a radiação solar de onda curta incidente na superfície 
(Downward Shortwave Radiation at Surface), que representa o total de energia solar 
disponível após atenuação atmosférica, com máximos superiores a 800 W m$^{-2}$ em 
regiões de baixa cobertura de nuvens e aerossóis; e (iii) a radiação solar de onda 
curta refletida no topo da atmosfera (Reflected Shortwave Radiation at TOA), indicador 
do albedo planetário, com valores elevados (superiores a 400 W m$^{-2}$) sobre regiões 
de alta refletividade como nuvens, gelo e áreas desérticas.

\section{Revisão de Estudos Anteriores}

Estudos recentes têm apresentado evidências consistentes de mudanças nos padrões climáticos em diferentes regiões do Brasil, incluindo alterações na temperatura do ar, na precipitação e na disponibilidade de radiação solar à superfície. No entanto, a maior parte dessas investigações concentra-se em variáveis específicas ou em regiões isoladas, sendo ainda relativamente escassos os trabalhos que abordam simultaneamente múltiplas variáveis climáticas sob uma perspectiva comparativa regional, especialmente considerando diferentes regimes climáticos e fontes de dados observacionais e de sensoriamento remoto.

Na região Nordeste do Brasil, diversos estudos têm investigado a variabilidade climática e a ocorrência de eventos extremos associados à precipitação e à temperatura. \cite{marengo2017} analisaram a seca prolongada ocorrida entre 2012 e 2017, considerada a mais severa desde a década de 1960, demonstrando que o evento resultou não apenas da redução da precipitação, mas também do aumento das temperaturas médias, que intensificou a evapotranspiração e agravou os impactos hidrológicos e socioeconômicos. 

Outros trabalhos apontam que a combinação entre aquecimento regional e alterações nos padrões de circulação atmosférica tem ampliado a vulnerabilidade do semiárido nordestino a eventos extremos de seca, com implicações diretas sobre a disponibilidade hídrica e a segurança alimentar.

Na região Sudeste, estudos têm documentado tanto alterações nos regimes de precipitação quanto tendências de aumento da temperatura do ar, com impactos significativos sobre os recursos hídricos e os sistemas urbanos. \cite{coelho2016} realizaram um diagnóstico detalhado do evento seco excepcional associado ao verão austral de 2013--2014 no estado de São Paulo, integrando análises espaciais e temporais da precipitação. 

A partir da comparação entre a climatologia mensal do período de 1981--2010 e as observações dos períodos de outubro de 2013 a março de 2014 e de outubro de 2014 a março de 2015, os autores evidenciaram reduções expressivas nos totais mensais de precipitação ao longo da estação chuvosa, com valores sistematicamente inferiores à média climatológica. A análise da precipitação acumulada anual indicou que o ano hidrológico de 2014 apresentou totais excepcionalmente baixos durante todo o período chuvoso, posicionando-se próximos aos limites inferiores da variabilidade climatológica, caracterizando o verão de 2013--2014 como o mais seco da série analisada.

\begin{figure}[H]
\centering
\includegraphics[width=0.6\textwidth]{fig/image.png}
\caption[Distribuição de precipitação observada durante os períodos chuvosos.]
{(a) Distribuição mensal da precipitação observada durante os períodos chuvosos de 2013--2014 e 2014--2015, em comparação com a climatologia de 1981--2010. (b) Precipitação acumulada, mostrando que o ano hidrológico de 2014 apresentou valores próximos aos limites inferiores da variabilidade climatológica. Extraído de \cite{coelho2016}}
\label{fig:coelho_boxplot_cumul}
\end{figure}

A análise da precipitação acumulada anual demonstra que o déficit pluviométrico observado não se restringiu a episódios isolados, mas refletiu uma persistência temporal da redução das chuvas ao longo da estação chuvosa. Esses resultados reforçam a caracterização do verão de 2013--2014 como um evento extremo, associado à variabilidade climática natural e a anomalias nos padrões de circulação atmosférica que reduziram a atuação dos sistemas responsáveis pela precipitação no Sudeste do Brasil \citep{coelho2016}.

Além dos eventos hidrológicos extremos, estudos recentes têm destacado o papel do aquecimento regional e da urbanização na modulação do clima local. No contexto urbano, diversas pesquisas têm evidenciado a intensificação das ilhas de calor urbanas em grandes regiões metropolitanas brasileiras, associada à expansão da área construída, à redução da cobertura vegetal e às alterações nas propriedades radiativas da superfície \citep{Campelo, Almeida}. Esses processos contribuem para o aumento das temperaturas locais e podem interagir com os regimes de precipitação, intensificando impactos térmicos e hidrológicos.

No que se refere à radiação solar incidente na superfície, produtos derivados de 
sensoriamento remoto têm sido amplamente empregados para complementar redes observacionais com cobertura espacial limitada. Para São Paulo, \citet{yamasoe2017} analisaram 11 anos (2005--2015) de medidas de irradiância solar global descendente e estimaram os efeitos radiativos climatológicos de aerossóis e nuvens sobre a superfície. Os autores utilizaram o código de transferência radiativa LibRadtran para separar o efeito direto dos aerossóis do efeito das nuvens, demonstrando que, climatologicamente, as nuvens podem ser até 4 vezes mais efetivas que os aerossóis na atenuação da radiação solar. O efeito radiativo de onda curta das nuvens apresentou redução máxima de aproximadamente 170~W/m$^{2}$ em janeiro e mínima de 37~W/m$^{2}$ em julho, enquanto o efeito direto dos aerossóis foi máximo na primavera, atingindo cerca de 50~W/m$^{2}$ em meados de setembro devido ao transporte de fumaça de queimadas da Amazônia. Durante o restante do ano, o efeito médio dos aerossóis foi de aproximadamente 20~W/m$^{2}$, atribuído a fontes urbanas locais.

Para o Rio Grande do Norte, \citet{medeiros2017} realizaram a calibração da equação de Ångström-Prescott\footnote{A equação de Ångström-Prescott relaciona a radiação solar global incidente na superfície ($H$) com a radiação solar no topo da atmosfera ($H_0$) e a duração de insolação ($n$) relativa ao fotoperíodo ($N$), sendo expressa por: $H/H_0 = a + b(n/N)$, onde $a$ e $b$ são coeficientes empíricos que devem ser calibrados localmente para representar adequadamente as condições atmosféricas da região. O coeficiente 
$a$ representa fisicamente a fração de radiação difusa em relação a $H_0$ em dias completamente 
nublados, enquanto $b$ quantifica a fração adicional de radiação transmitida em função da duração de insolação.} para estimativa de radiação solar diária em quatro estações meteorológicas do estado (Natal, Macau, Apodi e Caicó) durante o período de 2008--2013. Os autores reportaram que mais de 90\% da área do estado é classificada como semiárida. O desempenho da equação calibrada foi considerado satisfatório, com erro médio (MBE) inferior a 1,50~MJ~m$^{-2}$~dia$^{-1}$, coeficiente de correlação de Pearson em torno de 0,90 e índice de concordância de Willmott superior a 0,90. Os resultados demonstraram que a calibração local dos coeficientes da equação de Ångström-Prescott é essencial, uma vez que os valores obtidos diferiram significativamente dos valores recomendados pela FAO, refletindo as características atmosféricas específicas da região semiárida.

Em escala temporal mais longa, \citet{yamasoe2021} analisaram 56 anos (1961--2016) de irradiação solar na superfície, duração de insolação, amplitude térmica diurna e fração de cobertura de nuvens em São Paulo. Os autores identificaram uma tendência negativa (\textit{dimming}) de 1961 até o início dos anos 1980, consistente com o observado em outras partes do mundo. Entretanto, diferentemente do \textit{brightening} observado em países industrializados ocidentais nas décadas recentes, a tendência negativa persistiu em São Paulo, com taxa de $-0,13$~MJ~m$^{-2}$ por década ($p = 0,006$).

Produtos de irradiância solar derivados de satélites geoestacionários, como o GOES, e de reanálises globais, como o ERA5-Land, representam ferramentas essenciais para estudos regionais de variabilidade e tendências climáticas, permitindo análises consistentes em diferentes escalas temporais e espaciais. No entanto, ainda são limitados os estudos que integram de forma sistemática dados observacionais e produtos de satélite e reanálise para a análise conjunta de temperatura, precipitação e radiação solar em diferentes regiões do 
Brasil.

Diante desse cenário, torna-se evidente a necessidade de estudos que abordem simultaneamente múltiplas variáveis climáticas, considerando diferentes regimes climáticos e combinando dados observacionais e produtos de sensoriamento remoto. Essa abordagem integrada é fundamental para ampliar a compreensão da variabilidade climática regional e subsidiar avaliações mais robustas sobre tendências e impactos associados às mudanças climáticas no território brasileiro.
\newpage
\section{Objetivos}

\subsection{Objetivo Geral}

Analisar a variação temporal de variáveis meteorológicas nos estados de São Paulo e Rio Grande do Norte, utilizando dados observacionais e produtos de satélite e reanálise, visando identificar tendências climáticas no período de 1990 a 2025.

\subsection{Objetivos Específicos}

\begin{itemize}
\item Caracterizar os padrões de precipitação, temperatura e radiação solar incidente 
  em superfície utilizando dados do Instituto Nacional de Meteorologia (INMET), Agência Nacional de Águas e Saneamento Básico (ANA), Climate Hazards Group InfraRed Precipitation with Station data (CHIRPS), Geostationary Operational Environmental Satellite (GOES) e European ReAnalysis 5th Generation (ERA5-Land);
\item Avaliar a qualidade dos dados por meio de validação entre fontes observacionais e produtos de satélite e reanálise;
\item Identificar e quantificar tendências climáticas por meio de métodos estatísticos;
\item Determinar as normais climatológicas e avaliar desvios em relação aos valores de referência;
\item Investigar a relação entre tendências climáticas e cobertura da superfície com dados do MapBiomas;
\item Comparar os padrões de tendências climáticas entre São Paulo e Rio Grande do Norte.
\end{itemize}
