\chapter{Conclusões}
\label{cap:conclusoes}

Este trabalho de dissertação investigou as tendências de temperatura do ar, precipitação acumulada e radiação solar incidente nos estados de São Paulo e Rio Grande do Norte durante o período de 1990 a 2024, considerando dados observacionais de estações meteorológicas com produtos de satélite e reanálise. A análise combinou validação cruzada entre fontes, detecção de tendências por meio do teste de Mann-Kendall e estimador de Sen, caracterização de normais climatológicas e avaliação quantitativa da influência da cobertura da terra sobre as variáveis climáticas.

\section{Tendências Climáticas Observadas}

Os resultados demonstram aquecimento generalizado em ambos os estados, embora com diferenças na magnitude e distribuição espacial. São Paulo apresentou um comportamento de aquecimento acumulado da ordem de 1{,}0 a 1{,}5 °C ao longo das três décadas e meia analisadas, com tendências estatisticamente significativas em praticamente todo o território. A primavera destacou-se como a estação com aquecimento mais intenso e espacialmente homogêneo, com taxas frequentemente superiores a 0{,}05 °C/ano.

No Rio Grande do Norte, o aquecimento médio situou-se entre 0{,}7 e 1{,}1 °C, valores ligeiramente inferiores aos observados no estão de São Paulo, mas ocorrendo em contexto onde as temperaturas absolutas já são substancialmente mais elevadas.

A precipitação apresentou padrões substancialmente mais complexos e espacialmente heterogêneos em ambos os estados. São Paulo exibiu tendências predominantemente negativas, especialmente durante outono e primavera em localidades do interior, embora a maioria dessas tendências não tenha alcançado significância estatística devido à alta variabilidade interanual. No Rio Grande do Norte, a variabilidade climática extrema, com coeficientes de variação frequentemente excedendo 40 a 50\%, mascarou possíveis sinais de mudança sistemática, impossibilitando conclusões definitivas sobre alterações nos regimes de precipitação acumulada.

A radiação solar incidente demonstrou tendências predominantemente positivas em São Paulo, particularmente intensas durante o inverno quando a menor nebulosidade facilita maior transmissão atmosférica. No Rio Grande do Norte, os padrões foram mais heterogêneos, com algumas regiões apresentando incrementos significativos enquanto outras mostraram estabilidade ou reduções localizadas.

\section{Validação de Produtos de Satélite e Reanálise}

A validação via correlação entre dados observacionais e produtos de satélite e demonstrou concordância variável conforme a variável e região consideradas. A temperatura apresentou concordância com os dados do ERA5-Land ($R \geq 0{,}9$), sem vieses sistemáticos significativos, garantindo alta confiabilidade às análises de tendências. A radiação solar mostrou concordância boa a muito boa ($R = 0{,}6$ a $0{,}9$), com o produto CLARA-A3 demonstrando desempenho superior ao GOES-16 em São Paulo, embora apresentasse maior dispersão residual.

A precipitação, apesar de correlações moderadas a boas entre dados da ANA e CHIRPS ($R = 0{,}7$ a $0{,}9$), apresentou incertezas substancialmente maiores. O CHIRPS sistematicamente subestimou acumulados elevados em São Paulo (coeficiente angular de 0{,}544), comportamento frequentemente observado em produtos de satélite quando confrontados com eventos convectivos intensos. No Rio Grande do Norte, o desempenho foi superior ($R = 0{,}9$), refletindo a natureza distinta dos regimes de precipitação dominantes..

\section{Influência da Cobertura da Terra sobre o Clima Local}

A análise quantitativa da relação entre cobertura da terra e variáveis climáticas revelou que a urbanização constitui a principal forçante climática detectável em escala local. Em São Paulo, a relação entre área urbanizada e temperatura apresentou a maior significância estatística de toda a análise (93{,}8\% dos pontos), com sensibilidade mediana de 0{,}97 °C por ponto percentual de urbanização. Este resultado implica que áreas com urbanização intensa (30 a 50\% de cobertura urbana) podem apresentar aquecimento atribuível à urbanização da ordem de 3 a 5 °C.

A detecção do efeito urbano mostrou-se fortemente modulada pelo contexto climático regional. Enquanto em São Paulo praticamente todos os pontos apresentaram relação significativa entre urbanização e temperatura, no Rio Grande do Norte apenas 2{,}0\% alcançaram significância estatística. Esta diferença reflete mecanismos biofísicos fundamentais: em climas úmidos, a vegetação rural transpira ativamente e resfria o ar circundante, fazendo com que sua substituição por superfícies impermeáveis resulte em forte aquecimento. Em climas secos e semiáridos, a vegetação rural já apresenta evapotranspiração limitada devido ao estresse hídrico, reduzindo o diferencial térmico potencial entre áreas urbanas e rurais.

As formações vegetais nativas apresentaram relações substancialmente mais fracas e espacialmente inconsistentes com as variáveis climáticas. A ausência de relação detectável entre cobertura de Caatinga e temperatura no Rio Grande do Norte (apenas 0{,}6\% de significância) contrasta com a literatura que documenta efeitos de resfriamento da vegetação em escala global, mas é consistente com as características da vegetação semiárida. Em condições de déficit hídrico severo característico de 8 a 9 meses do ano no semiárido, a Caatinga adota estratégias de conservação de água que minimizam a transpiração e consequentemente o efeito de resfriamento evaporativo.

Em São Paulo, as correlações observadas entre cobertura de Formação Florestal e variáveis climáticas refletem primariamente a distribuição geográfica das florestas remanescentes, historicamente preservadas em áreas de topografia acidentada na Serra do Mar e Serra da Mantiqueira, ao invés de efeitos causais diretos da vegetação.

\section{Transformações no Uso e Cobertura da Terra}

As análises baseadas em dados do MapBiomas descreveram transformações substanciais na cobertura da terra em ambos os estados durante o período 1990--2020. São Paulo sofreu uma conversão massiva de pastagens de baixa produtividade (redução de 57{,}4\% para 7{,}8\% do território) em agricultura intensiva mecanizada, principalmente cana-de-açúcar, simultaneamente a processo de recuperação florestal em áreas protegidas da Serra do Mar.

O Rio Grande do Norte apresentou uma das conversões de vegetação nativa mais severas quando comparada primeiramente ao estado de São Paulo, com redução da cobertura de Caatinga de aproximadamente 79\% para 35{,}5\% em células de grade com maior pressão antrópica. Esta degradação concentrou-se espacialmente nas regiões centro-norte e central, sendo particularmente acelerada na última década (2010--2020), quando ocorreu perda de 24{,}5 pontos percentuais em apenas 10 anos.

\section{Limitações e Incertezas}

Este estudo apresenta limitações que devem ser consideradas na interpretação dos resultados. A série temporal de 35 anos, embora substancial, pode ser insuficiente para distinguir claramente tendências de longo prazo associadas às mudanças climáticas antropogênicas de flutuações naturais de baixa frequência, particularmente em regiões com alta variabilidade climática interanual como o semiárido nordestino.

A abordagem de regressão linear entre cobertura da terra e variáveis climáticas, embora amplamente utilizada na literatura, não estabelece causalidade definitiva devido à possibilidade de fatores confundidores espacialmente correlacionados. Análises mais aprofundadas, incluindo experimentos de modelagem numérica, seriam necessárias para estabelecer relações causais significativas.

\section{Implicações e Recomendações}

Os resultados deste trabalho têm implicações importantes para planejamento urbano, gestão de recursos hídricos e adaptação às mudanças climáticas. A magnitude do efeito de ilha de calor urbana demonstradas em São Paulo sugere que estratégias de mitigação baseadas em aumento de cobertura vegetal urbana e modificação de propriedades de superfície podem ter impactos significativos sobre o conforto térmico e a demanda energética.

No Rio Grande do Norte, a degradação massiva da Caatinga, combinada com o aquecimento já observado e projetado para continuar, configura cenário de vulnerabilidade amplificada. A preservação e recuperação da vegetação nativa, além de seu valor intrínseco para conservação da biodiversidade, pode contribuir para manutenção de serviços ecossistêmicos essenciais em contexto de mudanças climáticas, já que segundo o \cite{sudene2024desertificacao}, foi feito um diagnóstico que apresenta um valor de 95\% do território do Rio Grande do Norte estar suscetível a desertificação, cenário este que vai de encontro com os resultados aqui apresentados.

\section{Perspectivas para Trabalhos Futuros}

Este trabalho abre várias perspectivas para investigações futuras. Análises com resolução temporal mais fina (diária ou horária) poderiam revelar padrões não detectáveis em médias sazonais, particularmente no que se refere aos efeitos da vegetação sobre o clima local.

Experimentos de modelagem numérica com modelos climáticos regionais, nos quais cenários de uso da terra são sistematicamente variados mantendo-se outras condições constantes, forneceriam evidências sobre relações causais entre cobertura da terra e clima. Tais experimentos poderiam também avaliar efeitos não-lineares e interações entre diferentes tipos de cobertura que não são adequadamente capturados por análises de regressão linear.

A extensão das análises para outras regiões do Brasil com regimes climáticos distintos permitiria avaliação mais abrangente da generalidade dos padrões aqui documentados. Particularmente interessante seria a análise de regiões de transição entre biomas, onde mudanças na cobertura vegetal podem ter efeitos mais pronunciados sobre o clima local.

Finalmente, a integração das tendências climáticas documentadas com projeções de modelos climáticos globais e regionais para cenários futuros permitiria avaliação mais abrangente dos riscos e vulnerabilidades associados às mudanças climáticas projetadas.