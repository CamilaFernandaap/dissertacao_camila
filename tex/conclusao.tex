\chapter{Conclusões}
\label{cap:conclusoes}

Este trabalho investigou de forma integrada as tendências de temperatura do ar, precipitação acumulada e radiação solar incidente nos estados de São Paulo e Rio Grande do Norte durante o período de 1990 a 2024, combinando dados observacionais de estações meteorológicas (2004–2024) com produtos de satélite e reanálise (1990–2024), além de avaliar quantitativamente a influência das mudanças no uso e cobertura da terra com base nos dados do MapBiomas. A abordagem adotada permitiu caracterizar não apenas a evolução temporal dessas variáveis climáticas, mas também compreender os mecanismos físicos associados às mudanças observadas, considerando tanto forçantes climáticas de grande escala quanto modificações locais da superfície.

A validação cruzada entre dados observacionais e produtos de satélite e reanálise demonstrou que a temperatura do ar apresentou elevada confiabilidade, com coeficientes de correlação superiores a $R \geq 0{,}9$ entre os dados observacionais e o ERA5-Land, evidenciando que este produto representa adequadamente a variabilidade térmica regional. A radiação solar incidente apresentou concordância boa a muito boa entre os dados observacionais e os produtos de satélite, com coeficientes de correlação variando entre aproximadamente $R = 0{,}6$ e $R = 0{,}9$, particularmente com o produto CLARA-A3, enquanto o GOES-16 apresentou maior dispersão, porém ainda dentro de limites aceitáveis para análises climatológicas. A precipitação apresentou maiores incertezas relativas, com correlações variando entre $R = 0{,}7$ e $R = 0{,}9$, refletindo a complexidade inerente à sua estimativa por sensoriamento remoto, especialmente em regiões com forte variabilidade espacial e temporal. Ainda assim, os níveis de correlação observados confirmam que os produtos utilizados são adequados para a análise de tendências climáticas em escala regional, fornecendo uma base robusta para as análises realizadas neste estudo.

Os resultados evidenciaram aquecimento generalizado em ambos os estados ao longo do período analisado, com magnitude substancialmente maior no estado de São Paulo, onde o aumento acumulado variou entre aproximadamente $0{,}6$ e $2{,}5$ °C. As tendências positivas de temperatura observadas na maior parte do território paulista indicam concordância com o aumento da radiação solar incidente e com alterações no balanço energético da superfície. Esse aquecimento foi espacialmente coerente e apresentou elevada significância estatística em grande parte dos pontos analisados, indicando que não se trata apenas de flutuações naturais de curto prazo, mas sim de uma tendência persistente ao longo das últimas três décadas e meia. No Rio Grande do Norte, embora o aquecimento observado tenha sido de menor magnitude, variando entre aproximadamente $0{,}3$ e $0{,}6$ °C, sua presença consistente em diferentes regiões do estado confirma que o sinal de aquecimento também está presente nesse contexto climático. A menor magnitude observada nesse estado pode ser atribuída, em parte, à maior influência da variabilidade climática natural característica de regiões semiáridas, que introduz maior dispersão estatística nas séries temporais e dificulta a detecção de tendências de longo prazo.

A análise da radiação solar incidente revelou tendências predominantemente positivas, especialmente no estado de São Paulo, com aumentos da ordem de dezenas de W/m$^2$ ao longo do período analisado, indicando aumento da energia radiativa disponível na superfície. Esse resultado é fisicamente consistente com o aquecimento observado, uma vez que a radiação solar constitui a principal fonte de energia do sistema climático terrestre e exerce controle direto sobre a temperatura da superfície. No Rio Grande do Norte, as tendências de radiação apresentaram perfil consistente e positivo em praticamente todo o estado, com destaque para a estação do outono, onde os valores apresentaram-se mais elevados que nas demais estações e demonstraram maior proporção de pontos com significância estatística. Apesar de esse resultado, a princípio, demonstrar discordância com o observado para a precipitação, a análise integrada das séries temporais de precipitação diária acumulada para o centro-extremo oeste do estado indicou que a precipitação ocorre predominantemente durante o período noturno, reduzindo a influência direta da nebulosidade associada à precipitação sobre os valores médios diurnos de radiação solar incidente.

A precipitação apresentou comportamento substancialmente mais complexo e variável em ambos os estados. Em São Paulo, observou-se predominância de tendências negativas em diversas regiões, particularmente no interior do estado, embora a maioria dessas tendências não tenha atingido significância estatística. Esse resultado indica que, embora possam estar ocorrendo alterações no regime pluviométrico, fatores orográficos e a variabilidade natural do sistema climático podem estar diretamente relacionados ao perfil espacial observado, tornando mais difícil a identificação de um padrão homogêneo. No Rio Grande do Norte, a elevada variabilidade interanual da precipitação dificultou a identificação de tendências robustas, com poucos pontos apresentando significância estatística, embora tenham sido observadas tendências positivas mais concentradas no outono, especialmente nas regiões central e extremo oeste do estado. Parte dessa região corresponde aos brejos de altitude, que contribuem para um perfil climático diferenciado, com temperaturas relativamente mais amenas e maior disponibilidade hídrica local.

A análise da influência da cobertura da terra revelou que as modificações antrópicas da superfície desempenham papel significativo na modulação do clima local, particularmente no estado de São Paulo. A forte associação estatística observada entre áreas urbanizadas e aumento da temperatura, com coeficiente mediano de regressão de aproximadamente $\beta_{\mathrm{med}} = 4{,}75$ e significância estatística em cerca de $93{,}8\%$ dos pontos analisados, constitui evidência clara da presença do efeito de ilha de calor urbana. Esse fenômeno resulta da substituição da cobertura vegetal por superfícies impermeáveis, que apresentam maior capacidade de absorção e armazenamento de energia térmica e menor capacidade de dissipação de calor por evapotranspiração.

Inicialmente, identificou-se uma redução significativa nas vegetações nativas e áreas de pastagem nos dois estados, em especial em São Paulo, onde as áreas de pastagem apresentaram redução de aproximadamente $15{,}3$ pontos percentuais (de $37{,}4\%$ para $22{,}1\%$), enquanto áreas agrícolas, como o cultivo de cana-de-açúcar, apresentaram aumento de aproximadamente $8{,}6$ pontos percentuais (de $3{,}5\%$ para $12{,}1\%$). No Rio Grande do Norte, a cobertura de formação savânica apresentou redução de aproximadamente $3{,}4$ pontos percentuais (de $27{,}0\%$ para $23{,}6\%$), que, embora aparentemente pequena em termos percentuais, representa uma alteração significativa considerando a extensão territorial do estado e a importância dessa cobertura para o equilíbrio climático regional.

Ao observar a influência da cobertura vegetal natural, os resultados indicaram limitação e forte dependência do contexto climático regional. Em São Paulo, áreas com maior cobertura florestal apresentaram tendência a temperaturas relativamente mais baixas, refletindo o papel da vegetação na regulação térmica por meio da evapotranspiração e do sombreamento da superfície. No Rio Grande do Norte, no entanto, esse efeito foi significativamente mais fraco, com significância estatística inferior a $1\%$ em diversas análises, o que pode ser explicado pelas características ecofisiológicas da vegetação da Caatinga, que apresenta evapotranspiração naturalmente limitada devido à disponibilidade restrita de água no solo.

Ao passar para a quantificação direta da sensibilidade das variáveis climáticas às mudanças na cobertura da superfície, por meio da análise dos coeficientes de regressão, os resultados demonstraram que a urbanização constitui o fator mais fortemente associado às alterações térmicas em escala local. Em São Paulo, os coeficientes de regressão apresentaram valores elevados e positivos, com significância estatística superior a $90\%$ dos pontos analisados. Essa forte associação quantitativa evidencia que o aumento da urbanização promove modificações sistemáticas no balanço energético da superfície, resultando em aumento da temperatura do ar.

Em contraste, no estado do Rio Grande do Norte, a relação entre urbanização e temperatura apresentou magnitude substancialmente menor, com coeficientes medianos inferiores a $\beta_{\mathrm{med}} = 1{,}0$ e significância estatística inferior a $2\%$, indicando que a influência da urbanização sobre o clima local é fortemente modulada pelo regime climático regional. Esse resultado reflete a limitação hídrica característica de regiões semiáridas, onde a vegetação natural já apresenta evapotranspiração reduzida, diminuindo o contraste térmico entre superfícies vegetadas e urbanizadas.

De modo geral, a análise integrada dos resultados demonstrou que as mudanças observadas nas variáveis climáticas são resultado da interação entre múltiplos fatores atuando em diferentes escalas espaciais e temporais. O aquecimento observado em ambos os estados é consistente com o sinal de mudanças climáticas globais. Ao mesmo tempo, as modificações locais no uso e cobertura da terra atuam como moduladores regionais dessas tendências, amplificando ou atenuando seus efeitos dependendo das características da superfície e do contexto climático regional.

Os resultados obtidos neste estudo contribuem para o avanço do conhecimento sobre as mudanças climáticas regionais no Brasil, demonstrando que o clima responde não apenas às forçantes globais associadas às mudanças na composição atmosférica, mas também às transformações locais da superfície terrestre. A urbanização emerge como um fator particularmente relevante na modificação do clima local, especialmente em regiões com maior disponibilidade hídrica.

Do ponto de vista científico, este trabalho demonstra a importância da integração entre diferentes fontes de dados e abordagens metodológicas na análise de tendências climáticas regionais. A utilização combinada de dados observacionais, produtos de satélite e reanálise permitiu superar limitações associadas a cada fonte individualmente, fornecendo uma caracterização mais completa das mudanças observadas.

De forma geral, os resultados apresentados evidenciam que o sistema climático regional nos estados de São Paulo e Rio Grande do Norte está passando por modificações consistentes com o aquecimento observado em escala global, ao mesmo tempo em que responde às transformações locais associadas às mudanças no uso e cobertura da terra. Esses resultados destacam a importância de considerar tanto as forçantes climáticas globais quanto os processos locais na compreensão das mudanças climáticas regionais.