Agradeço, primeiramente, à minha orientadora, Márcia Akemi, que desde a graduação até hoje, no final da minha trajetória como mestranda, vem me mostrando constantemente a sua força como educadora, cientista e pessoa. Com ela pude aprender a importância de ir atrás das respostas e não apenas aceitá-las. Tive e tenho muita sorte em tê-la como orientadora deste trabalho; foi um desafio mútuo, muitos resultados foram surpreendentes para ambas, mas com seu auxílio constante, cada desafio foi capaz de ser vencido;
	
À minha família, em especial aos meus pais, Maria e Vicente, a quem dedico cada gota de suor derramada para que eu pudesse estar aqui. Mesmo com todas as atribulações da vida, vocês nunca me abandonaram;

Ao meu marido, Eduardo, que foi meu alicerce desde o início da minha vida acadêmica, quem segurou a minha mão a cada desafio, a cada ``não'', enfim, em todos os momentos desde que nos conhecemos;

Ao meu grupo de pesquisa, LRAA, com o qual pude aprender muito, expressar as minhas dúvidas e receber dicas valiosas, tornando esta trajetória mais suave;

Aos professores Tércio, Rose e Rita, vocês, além de excelentes professores e exemplos do que são como pessoas e profissionais, me deram sugestões valiosíssimas sobre a melhoria do meu trabalho, sobre como ter segurança no momento de cada apresentação e sobre a importância do tratamento dos dados;

A toda equipe da pós-graduação, que sempre foi imensamente solícita e que, mesmo com os inúmeros e-mails, nunca deixou de ser gentil e me auxiliou da melhor forma possível;

Ao IAG e à CAPES pelo apoio financeiro entre setembro de 2024 e agosto de 2026 (CAPES-PROEX 88887.999930/2024-00); ao INCT Klimapolis por me proporcionar a oportunidade de contribuir com os resultados aqui apresentados.
